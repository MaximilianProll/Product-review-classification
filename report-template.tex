\documentclass[a4paper,11pt]{article}

\begin{document}

\title{Title of the Mini Project\\
\normalsize (group number and name here)}

\author{First Author (student ID) \and Second Author (student ID)
  \and Third Author (student ID)}

\maketitle

\newcommand{\points}[1]{\par\noindent\textit{(#1 points)}}
\newcommand{\onepoint}{\par\noindent\textit{(1 point)}}

\begin{abstract}
  Write an abstract of approximately 200 words.

  The overall length of your report in this format should be 6--8
  pages, including images, references and everything, in the format of
  this template.  You can choose to use any document template and
  typesetting facility as long as you submit your report as a PDF file.
  Use of figures and tables is strongly encouraged!

  The breakdown of the total 30 points of the mini project is shown
  below for all sections of the report. However, points will be given
  only for project reports that are complete, i.e.\@ they have
  relevant content in all sections.  Minimum of 10 points are required
  for passing the course.  
  
  The reports will be evaluated using the Turnitin plagiarism
  prevention tool to ensure that the reports are genuine work written
  for this course.
  %% 
  \onepoint
\end{abstract}

\section{Introduction}

Describe the machine learning problem you have addressed in your mini
project, your applied method and used data in general terms.
%%
\points{2}

\section{Related work}

Give a brief literature survey on the prior works and state of the art
in the problem setting and the methods people have earlier applied for
solving it.
%%
\points{2}

\section{Method}

Describe your chosen solution in sufficient detail.  Pay special
attention to describing the deep learning algorithm(s) you used.  Show
some essential equations and describe all hyperparameters that are
involved in the method.
%%
\points{3}

\section{Data}

Tell where the data is from, what it contains, the number of samples
in training, validation and test sets, the dimensional, etc.  Also
describe the preprocessing stages applied by the providers of the data
or by you yourselves.
%%
\points{3}

\section{Experiments}

Explain the goal and implementation of your experiments.  Tell what
hyperparameters values you experimented with and what other
variations in the method you tested.
%%
\points{5}

\section{Results}

Give the results of your experiments in a table and explain them in
the text.  Some figures would be good here for illustration.  Refer to
the table(s) and image(s) in the text and describe them.
%%
\points{5}

\section{Discussion}

Discuss your results in comparison with comparable results your have
found in the literature and web.  Explain any other findings you made
while running the experiments.
%%
\points{4}

\section{Conclusions}

Give your final conclusions from the whole mini project and its results.
%%
\points{3}

\section{References}

Most likely between 5--10 references to related works and results.
%%
\points{2}

\section{Roles of the authors}

If you have more than one member in your mini project group, you need
to explain in this section how the labor was divided between you and
what were each one's roles in the project and its reporting.

\end{document}